\documentclass[11pt]{article}
    
\usepackage{amsmath}
\usepackage{amsfonts}
\usepackage{cite}
\usepackage[letterpaper, total={6.5in, 9in}]{geometry}
\usepackage{mathpazo}
\usepackage{sourcecodepro}

\usepackage{hyperref}

\newcommand{\setcomp}[2]{\left\{ #1 \ \Big|\ #2 \right\}}
\newcommand{\rngto}[1]{1{:}#1}
\newcommand{\abs}[1]{\left| #1 \right|}
\newcommand{\absdet}[1]{\abs{#1}}
\newcommand{\dv}[1]{\mathrm{d}{#1}}
\newcommand{\Exp}{\mathrm{Exp}}
\newcommand{\vect}{\mathrm{vec}}
\newcommand{\vectu}{\mathrm{vecu}}


\begin{document}


\title{Efficient Unconstraining
  Parameter Transforms for Hamiltonian Monte Carlo}
\author{Meenal Jhajaria \\ \small Flatiron Institute \and Seth Axen \\
  \small University of T\"ubingen \and Bob
  Carpenter \\ \small Flatiron Institute}
\date{DRAFT: \today}
\maketitle


\begin{abstract}
  \noindent
  This paper evaluates the the statistical and computational
  efficiency of unconstraining parameter transforms for Hamiltonian
  Monte Carlo sampling.
\end{abstract}

\section{Introduction}

In statistical computing, we often need to compute high-dimensional
integrals over densities $\pi(x)$ (e.g., Bayesian estimation or
prediction, $p$-value calcuations, etc.).  The only black-box
techniques that work for general high-dimensional integrals are
Markov chain Monte Carlo (MCMC) methods.  The most effective MCMC
method in high dimensions is Hamiltonian Monte Carlo (HMC).  HMC works
by simulating the Hamiltonian dynamics of a fictitious particle
representing the value being sampled coupled with a momentum term.

Although it's possible to write HMC samplers that work for simply
constrained values such as upper- and/or lower-bounds
\cite{neal2011mcmc} or unit vectors \cite{byrne2013geodesic}, it is
far more challenging to do the same for complex constraints such as
simplices or positive definite matrices or for densities involving
multiple constrained values.  Instead, usually constrained values are mapped to unconstrained values before sampling. These transform mappings are formed out of the constraints on the model parameters. A change of variables adjustment is made to the target density for these transforms to work. The tricky part is- there can be more than one suitable mapping for every constraint. This leads us to the issue of selecting a transform. 

We evaluate the efficiency of unconstraining parameter transforms using metrics like effective sample size, leapfrog steps and squared error. Different transforms work well for different models, this paper aims to provide a useful analysis of the same, so practitioners can choose transforms that work well for their models.

\section{Transforms}

To draw samples from a distribution $p_X(x)$, defined on a constrained space $\mathcal{X}$ that can be uniquely parameterized by $n$ real degrees of freedom, we instead perform sampling in an unconstrained space $\mathcal{Y}=\mathbb{R}^m$ for $m \ge n$.

Let $f\colon \mathcal{Y} \to \mathcal{X}$ be a smooth and continuous map.
Only when $m = n$ can $f$ be bijective.
When $m > n$, multiple values of $y \in \mathcal{Y}$ may map to a given value of $x = f(y)$.
In this case, to make the function bijective, we consider an additional smooth and continuous map $g\colon \mathcal{Y} \to \mathcal{Z}$, where $\mathcal{Z}$ can be uniquely parameterized by $m - n$ real degrees of freedom.
We then define $h\colon \mathcal{Y} \to \mathcal{X} \times \mathcal{Z}: y \mapsto (f(y), g(y)) = (x, z)$.
$f$ and $g$ must be chosen so that $h$ is bijective almost-everywhere;
that is, the values of $y$ for which $h$ is non-bijective must have no probability mass.
When $h$ is non-bijective at single points, we call these singularities.
While singularities themselves are in general not a problem for sampling with MCMC, the region of a target density near the singularity tends to have high curvature.

\subsection{Changes of Variables}

If $X$ is a random variable with density $p_X$ and $Y = f(X)$ for a
smooth and bijective function $f$, then
\[
  p_Y(y) = p_X(f^{-1}(y)) \absdet{J_{f^{-1}}(y)},
\]
where the Jacobian of the inverse transform is defined by
\[
  J_{f^{-1}}(y) = \frac{\partial}{\partial y} \, f^{-1}(y)
\]
and
\[
  \absdet{J_{f^{-1}}(y)}
  = \abs{\det J_{f^{-1}}(y)}.
\]

\subsection{Target Density}
Given a proper joint distribution with a smooth and continuous density $p_{X,Y}(x, y | \theta)$ for some parameters $\theta$, we can use the usual change of variables formula to write

\[
  p_Y(y | \theta) = p_{X,Z}(f(y), g(y) | \theta) |J_h(y)|.
\]

To sample from a target density $p_X(x | \theta)$, let $p_{X,Z}(x, z | \theta) = p_X(x | \theta) p_Z(z | x, \theta)$.
Then we must choose some proper distribution $p_Z$ that is smooth and continuous.
Given that choice, we find that
\[
  p_Y(y | \theta) = p_X(f(y) | \theta) p_Z(g(y) | f(y), \theta) |J_h(y)|.
\]

Note that when $m > n$, it is insufficient to specify a transform $h$.
One must also select a prior $p_Z$, and it is desirable to establish some general properties that a given pair $h$ and $p_Z$ should have.
In addition to the above properties, the optimal pair would
1. discard as few degrees of freedom $|m - n|$ as possible.
2. produce a $p_Z$ and $|J_y|$ and their gradients that are efficient to compute.
3. produce a $p_Y$ that is easy to sample from.

We can expand a bit on the latter point.
As a general rule, the more normal $p_Y$ is, the easier it is to sample from (REF?).
Additionally, near singularities, small perturbations to $y$ tend to result in larger changes to $h(y)$ than far from the singularity.
This warps the volume around the singularity, which causes $|J_y(y)|$ to smoothly approach infinity near the singularity.
If $m$ is large, this is generally not a problem, as the "curse of dimensionality" works in our favor, and there is exponentially more volume away from the singularity than near the singularity.
However, for low $m$, it is necessary that $p_{X,Z}$ goes to 0 near the singularity faster than $|J_y(y)|$ goes to infinity.
While $p_X$ is generally fixed by the user-specified model and may not have this necessary property, the user may then choose a $p_Z$ that has this property.

\section{Preliminaries}

[SA: TODO: eliminate or merge with following section]

To compute Jacobians of functions with matrix-valued inputs or outputs, we need to implicitly or explicitly choose for the inputs and outputs a set of coordinates with dimension equal to the number of degrees of freedom of the matrix
Unless otherwise specified, we choose the coordinates to be the vectorization of the matrix.

We define the (bijective) vectorization of a matrix $X \in \mathbb{R}^{N \times K}$ with columns $x_k$ for $k \in \{1, ..., K\}$ as the vector
\[\vect(X) = \begin{bmatrix}x_1 \\ x_2 \\ \vdots \\ x_M\end{bmatrix}\]
formed by stacking the columns of $X$.

For $X \in \mathbb{R}^{N \times K}$, let $\vectu(X)$ be the map that stacks only the parts of the columns of $X$ above and including the diagonal.
This map is sometimes called the half-vectorization and is bijective for upper-triangular or symmetric $X$.

Similarly, $\vectu_-(X)$ extracts the same elements except the diagonal.
When $X$ is strictly upper-triangular or skew-symmetric (i.e. $X = -X^\top$), then this map is bijective.

\section{Computing Jacobian determinants}

[SA: TODO: remove everything we do not use later, add strategy for making non-square Jacobian square]

Here we introduce several strategies for computing Jacobian determinants.

The vectorization has the useful property that for all matrices $A$, $X$, and $B$, $\vect(AYB) = (B^\top \otimes A) \vect(Y)$, where $\otimes$ is the Kronecker product.
This allows us to convert expressions of matrices to expressions of vectors, from which the Jacobian can be read off.
For example, let $X = AYB$. Then $\dv{X} = A \dv{Y} B$, and $\vect(\dv{X}) = (B^\top \otimes A) \vect(\dv{Y})$, and the Jacobian is $J = B^\top \otimes A$.

When the previously discussed strategies are insufficient to compute the Jacobian, we bypass the Jacobian computation entirely and use the exterior product $\wedge$.
For example, let $X \in \mathbb{R}^{N \times K}$.
The exterior product
\[
  \dv{X_{11} }\wedge \dv{X_{21}} \wedge \ldots \wedge \dv{X_{NK}} = \bigwedge_{j=1}^K \bigwedge_{i=1}^N \dv{X}_{ij}
\]
is another more explicit way of writing the integration measure $\dv{X_{11}} \dv{X_{21}} \dots \dv{X_{NK}}$ that has useful properties for computing Jacobian determinants.
The basic properties we will use are

\begin{itemize}
  \item $a\dv{x} \wedge b\dv{y} = (ab)(\dv{x} \wedge \dv{y})$ for $a,b \in \mathbb{R}$
  \item $(\dv{x} + \dv{y}) \wedge \dv{z} = \dv{x} \wedge \dv{z} + \dv{y} \wedge \dv{z}$
  \item $\dv{x} \wedge \dv{y} = - \dv{y} \wedge \dv{x}$
  \item $\dv{x} \wedge \dv{x} = 0$
\end{itemize}

The connection between the exterior product and the Jacobian determinant is given by the following expression.
Given $f: y \mapsto x$ for $x,y \in \mathbb{R}^N$,
\[\bigwedge_{i=1}^N \dv{x_i} = \det(J_f(y)) \bigwedge_{i=1}^N \dv{y_i}.\]

We can then find the Jacobian determinant by first writing all differential forms $\dv{x_i}$ or $\dv{y_i}$ and then computing the exterior product until the expression looks like the above.
For our purposes, we will ignore the sign of the Jacobian determinant.

A related useful property is that for $X,Y,B \in \mathbb{R}^{N \times K}$, if $\dv{X} = B \dv{Y}$, then
\[\bigwedge_{j=1}^K \bigwedge_{i=1}^N \dv{X_i} = \det(B)^K \bigwedge_{j=1}^K \bigwedge_{i=1}^N \dv{Y_i}\]

Following [REF, Edelman], we use the notation $(\cdot)^\wedge$ as a shorthand to represent the exterior product of the unique elements of $\cdot$.

For example,
\begin{itemize}
  \item Rectangular matrix $\{X \in \mathbb{R}^{N \times K}\}$: $(\dv{X})^\wedge = \bigwedge_{j=1}^K \bigwedge_{i=1}^N \dv{X_{ij}}$
  \item Upper-triangular rectangular matrix $\{X \in \mathbb{R}^{N \times K} | X_{ij} = 0\ \forall\ i > j\}$: $(\dv{X})^\wedge = \bigwedge_{j=1}^K \bigwedge_{i=1}^{\min(j, N)} \dv{X_{ij}}$
  \item Diagonal matrix $\{X \in \mathbb{R}^{N \times K} | X_{ij} = 0\ \forall\ i \ne j\}$: $(\dv{X})^\wedge = \bigwedge_{i=1}^{\min(N, K)} \dv{X_{ii}}$
  \item Symmetric matrix $\{X \in \mathbb{R}^{N \times N} | X_{ij} = X_{ji}\}$: $(\dv{X})^\wedge = \bigwedge_{j=1}^K \bigwedge_{i=1}^{j} \dv{X_{ij}}$
  \item Skew-symmetric matrix $\{X \in \mathbb{R}^{N \times N} | X_{ij} = -X_{ji}, X_{ii} = 0\}$: $(\dv{X})^\wedge = \bigwedge_{j=1}^K \bigwedge_{i=1}^{j-1} \dv{X_{ij}}$
\end{itemize}

Using this notation, for example, $(B \dv{X})^\wedge = \det(B)^K (\dv{X})^\wedge$

\section{Unit simplex}

A unit $N$-simplex is an $N + 1$-dimensional vector of non-negative
values that sums to one.  As such, there are only $N$ degrees of
freedom, because if $x$ is an $N$-simplex, then
\[
  x_N = -(x_1 + x_2 + \cdots + x_{N-1}).
\]
Simplexes are useful for representations of multinomial probabilities
(e.g., probabilities of categories in a classification problem).

The set of unit $N$-simplexes is conventionally denoted
\[
  \Delta^N = \setcomp{x \in \mathbb{R}^{N + 1}}{\textrm{sum}(x) = 1
    \textrm{ and }
    x_n \geq 0 \textrm{ for } n \in \rngto{N}}
\]
Geometrically, an $N$-simplex is the convex closure of $N+1$ points
that are 1 in one coordinate and 0 elsewhere.  For example, the
3-simplex is the complex closure of
$\begin{bmatrix}1 & 0 & 0 \end{bmatrix},
\begin{bmatrix} 0 & 1 & 0 \end{bmatrix}$,
and $\begin{bmatrix} 0 & 0 & 1 \end{bmatrix}$.

\subsection{StickBreaking Transform}

The StickBreaking transform can be be understood from the stick-breaking construction for Dirichlet \cite{sethurman}. Intuitively, this comprises of recursively breaking a piece $x_i$ from a stick of unit length, where the leftover stick in the $i^{th}$ iteration is $ 1 - \sum_{1}^{i}x$. Let $y = f(x)$, then we define the stick-breaking mapping $ f \colon \Delta^{N-1} \xrightarrow{\makebox[0.4cm]{$\sim$}}  R^{N-1}$, for $1 \leq i \leq N$ as:	
\[
y_i
= \mathrm{logit}(z_i) - \mbox{log}\left(\frac{1}{N-i}
   \right) \text{for break proportion} \, 
   z_i = \frac{x_i}{1 - \sum_{i' = 1}^{i-1} x_{i'}}.
\]

The inverse transform $ f^{-1} \colon R^{N-1} \xrightarrow{\makebox[0.4cm]{$\sim$}}  \Delta^{N-1}$ is defined as:
\[
x_i =
\left( 1 - \sum_{i'=1}^{i-1} x_{i'} \right) \text{for break proportion} \, z_i = \mathrm{logit}^{-1} \left( y_i
                             + \log \left( \frac{1}{N - i}
                                            \right)\right).
                                            \]
                            
An $N$ dimensional unit simplex $\Delta^{N-1}$ has $N-1$ degrees of freedom (Notice that this inverse transform only maps the first $N-1$ elements, the last element of the simplex $x_{N} = 1 - \sum_1^{N-1}{x_i}$). The Jacobian matrix for $f^{-1}$ is a lower-triangular diagonal matrix. Much like the alr transform, for the change of variables we evaluate $\mathbf{J}_{i, i}$ where $i \in 1:N-1$.
\begin{align*}
\mathbf{J}_{i, i} &= \frac{\partial x_i}{\partial y_i}
=
\frac{\partial x_i}{\partial z_i} \,
\frac{\partial z_i}{\partial y_i}\\
\mathbf{J}_{i, i} &= \left(
  1 - \sum_{k' = 1}^{k-1} x_{k'}
   \right) z_k (1 - z_k),
\end{align*}

Absolute determinant of the diagonal matrix $\mathbf{J}$ is the product of its diagonal entries:
\begin{align*}
	\abs{\, det \, \textbf{J} \,} = \prod_{i=1}^{N-1} \textbf{J}_{i,i}
\end{align*}

The correction term $p_Y(y) = p_X(f^{-1}(y))\,
\prod_{i=1}^{N-1}z_i\,(1 - z_i)\left(1 - \sum_{i'=1}^{i-1} x_{i'}\right).$
\subsection{Additive log ratio transform}

The unconstraining transform for the identified softmax is known as
the additive log ratio (ALR) transform
\cite{aitchison1982statistical}, which is a bijection
$\textrm{alr}:\Delta^{N-1} \rightarrow \mathbb{R}^{N-1}$ defined for
$x \in \Delta^{N-1}$ by
\[
  \textrm{alr}(x)
  = \begin{bmatrix}\displaystyle
    \log \frac{x_1}{x_N} \cdots \log \frac{x_{N-1}}{x_N}
  \end{bmatrix}.
\]

The inverse additive log ratio transform maps values in
$\mathbb{R}^{N-1}$ to $\Delta^{N-1}$ defined for $y \in
\mathbb{R}^{N-1}$ by
\[
  \textrm{alr}^{-1}(y)
  = \textrm{softmax}(\begin{bmatrix} y &  0 \end{bmatrix}),
\]
where for $u \in \mathbb{R}^N$,
\[
  \textrm{softmax}(u) = \frac{\exp(u)}{\textrm{sum}(\exp(u))}.
\]

To calculate the change of variables adjustment, we consider only the
first $N-1$ coordinates of the result, because the last is defined in
terms of the first $N-1$.  For convenience, we define a function 
$s:\mathbb{R}^{N-1} \rightarrow \mathbb{R}^{N-1}$ that operates on
$N-1$ unconstrained variables and returns the first $N-1$ components
of the $\textrm{alr}^{-1}(y)$, which is
defined for $y \in \mathbb{R}^{N-1}$ by
\[
  s(y) = \frac{\exp(y)}{\textrm{sum}(\exp(y)) + 1}
  = \begin{bmatrix}
    \textrm{alr}^{-1}(y)_1
    & \cdots &
    \textrm{alr}^{-1}(y)_{N-1}
    \end{bmatrix}.
\]
Given a density $p_X(x)$ defined over simplices $x \in \Delta^{N-1}$,
we can transform to a density over unconstrained parameters $y \in
\mathbb{R}^{N-1}$ by applying the inverse ALR transform and adjusting
for the change of variables, which yields
\[
  p_Y(y) = p_X(\textrm{alr}^{-1}(y)) \absdet{J_{s}(y)},
\]
where $J_{s}(y)$ is the Jacobian of the function $s$ evaluated at $y$
and $\absdet{J_s(y)}$ is the absolute value of its determinant.


\subsubsection{Softmax Transform}

To calculate the determinant of the Jacobian of the inverse transform,
we start by noting that $s = \textrm{exp} \circ \textrm{norm}$, where
$\textrm{exp}$ is the elementwise exponential function and
\textrm{norm} is defined by
\[
  \textrm{norm}(z) = \frac{z}{\textrm{sum}(z) + 1}.
\]
As such, the resulting Jacobian determinant is the product of the
Jacobian determinants of the component functions,
\[
  \absdet{J_s(y)}
  = \absdet{J_{\textrm{exp}}(y)} \absdet{J_{\textrm{norm}}(z)},
\]
where $z = \textrm{exp}(y)$.  The Jacobian for the exponential
function is diagonal, so the determinant is the product of the
diagonal of the Jacobian, which for $y \in \mathbb{R}^{N-1}$ is
\[
  \absdet{J_{\textrm{exp}}(y)} = \textrm{prod}(\exp(y)).
\]
As above, let $z = \exp(y) \in (0, \inf)^{N-1}$.  We can differentiate
$\textrm{norm}$ to derive the Jacobian,
\[
  J_{\textrm{norm}}
  = \frac{1}{1 + \textrm{sum}(z)} \mathbb{I}_{N-1}
  - \left(\frac{1}{(1 + \textrm{sum}(z))^2} \beta \right)
  \textrm{vector}_{N-1}(1)^{\top},
\]
where $\mathbb{I}_{N-1}$ is the $(N - 1) \times (N - 1)$ unit matrix and
$\textrm{vector}_{N-1}(1)$ is the $N - 1$-vector with values 1.  Using
the matrix determinant lemma,\footnote{The matrix determinant lemma
  is \[\textrm{det}(A + u v^{\top}) = (1 + v^{\top} A^{-1} u)
    \textrm{det}(A).\]}
we have
\begin{eqnarray*}
  \textrm{absdet}(J_{\textrm{norm}}(z))
  & = &
  \left(
    1
    + \textrm{vector}_{N-1}(1)^{\top}
    \left(\frac{1}{1 + \textrm{sum}(z)} \mathbb{I} \right)^{-1}
    \frac{-z}{(1 + \textrm{sum}(z))^2}
    \right)
    \ \textrm{det}\left(\frac{1}{1 + \textrm{sum}(z)} \mathbb{I}
        \right)
  \\[6pt]
  & = & \left( \frac{1}{1 + \textrm{sum}(z)} \right)^N.
\end{eqnarray*}
Thus the entire absolute determinant of the Jacobian is defined by the
product, 
\[
  \absdet{J_s(y)}
  \ = \
  \textrm{prod}(\exp(y))
  \, \left( \frac{1}{1 + \textrm{sum}(\exp(y))} \right)^N.
\]
and our final expression for densities for unconstrained $y \in
\mathbb{R}^{N-1}$ is
\[
  p_Y(y)
  = p_X(\textrm{alr}^{-1}(y))
  \, \textrm{prod}(\exp(y))
  \left( \frac{1}{1 + \textrm{sum}(\exp(y))} \right)^N
\]  



\subsection{Simplex augmented softmax parameterization}

We define the transformation
$\phi: \mathbb{R}^n \to \Delta^{n-1} \times \mathbb{R}_{>0}: y \mapsto
(x_-, r)$, where $r = \sum_{i=1}^n e^{y_i}$ and
$x_i = \frac{1}{r} e^{y_i}$ for $1 \le i \le n-1$..

First we compute the scalar derivatives:
\[
\begin{aligned}
  \frac{\mathrm{d} r}{\mathrm{d} y_j}
  &= e^{y_j} = r x_j
  \\
  \frac{\mathrm{d} x_i}{\mathrm{d} y_j}
  &= \delta_{ij} \frac{1}{r} e^{y_i} - \frac{1}{r^2} e^{y_i} \frac{\mathrm{d} r}{\mathrm{d} y_j} = \delta_{ij} x_i - x_i x_j,
\end{aligned}
\]
which corresponds to the Jacobian
\[
  J = \begin{pmatrix}I_{n-1} - x_- \boldsymbol{1}_{n-1}^\top & -x_- \\
    r \boldsymbol{1}_{n-1}^\top & r \end{pmatrix} \mathrm{diag}(x).
\]

For invertible $A$, the determinant of the block matrix
$\begin{pmatrix}A & B \\ C & D\end{pmatrix}$ is $|A| |D-CA^{-1}B|$.  A
square matrix is invertible iff its determinant is non-zero.  From the
previous section,
\[
  |I_{n-1} - x_- \boldsymbol{1}_{n-1}^\top| = x_n > 0,
\]
so the determinant of the Jacobian is
\[
  |J| = x_n \left|r + r \boldsymbol{1}_{n-1}^\top (I_{n-1} - x_-
    \boldsymbol{1}_{n-1}^\top)^{-1} x_-\right|
  \prod_{i=1}^n x_i.
\]

Let $w = (I_{n-1} - x_- \boldsymbol{1}_{n-1}^\top)^{-1} x_-$. Then,
\[
\begin{aligned}
    w - x_- \sum_{i=1}^{n-1} w_i &= x_-\\
    w &= x_- \left(1 - \sum_{i=1}^{n-1} w_i\right)\\
    \sum_{i=1}^{n-1} w_i &= \sum_{i=1}^{n-1} \left( x_- (1 - \sum_{i=1}^{n-1} w_i) \right) = \left(\sum_{i=1}^{n-1} x_i \right) \left(1 - \sum_{i=1}^{n-1} w_i\right) = (1 - x_n)  \left(1 - \sum_{i=1}^{n-1} w_i\right)\\
    \sum_{i=1}^{n-1} w_i &= \frac{1 - x_n}{x_n} = \frac{1}{x_n} - 1\\
    w &= x_- \left(1 - \left(\frac{1}{x_n} - 1\right)\right) = \frac{1}{x_n} x_-
  \end{aligned}
\]

Then
\[
  |J| = x_n r \left|1 + \frac{1}{x_n}\sum_{i=1}^{n-1} x_i\right|
  \prod_{i=1}^n x_i = r \prod_{i=1}^n x_i
\]

To keep the target distribution proper, we must select a prior
distribution $\pi(r)$ for $r$.  If we choose $r \sim \chi_n$, then the
product of the correction and the density of the prior for $r$ is
proportional to

\[
  \mathrm{correction}
  = \pi(r) |J| = r^n e^{-r^2/2} \prod_{i=1}^n x_i
  = \exp\left(\sum_{i=1}^n y_i - \frac{1}{2}\left(\sum_{i=1}^n
      e^{y_i}\right)^2\right).
\]

Alternatively, if we choose $r \sim \mathrm{Gamma}(n, 1)$, then
\[
  \mathrm{correction} = \pi(r) |J| = r^n e^{-r} \prod_{i=1}^n x_i =
  \exp\left(\sum_{i=1}^n y_i - \sum_{i=1}^n e^{y_i}\right).
\]
This latter correction is equivalent to the sampling procedure from
the Dirichlet distribution with $\alpha_i=1$, where
$z_i \sim \mathrm{Exponential}(1)$ and
$y = \frac{z}{\sum_{i=1}^n z_i}$.

Both of these corrections can be captured with the generalization
\[
  \mathrm{correction}
  = \pi(r) |J|
  = r^n e^{-r^p/p} \prod_{i=1}^n x_i
  = \exp\left(\sum_{i=1}^n y_i - \frac{1}{p} \left(\sum_{i=1}^n e^{y_i}\right)^p\right),
\]
for $p > 0$, which corresponds to $r \sim \text{Generalized-Gamma}(1, n, p)$.

\subsection{Semi-orthogonal matrices (Stiefel manifold)}

A matrix $X \in \mathbb{R}^{N \times K}$ for $K \le N$ is semi-orthogonal if and only if $X^\top X = I_K$.
The semi-orthogonal matrices form a space called the Stiefel manifold.
When $K=N$, $X \in O(N)$ is an orthogonal matrix and also satisfies $X X^\top = I_N$ and $\det(X) = \pm 1$.
When $K=1$, $X$ is a point on the $(N-1)$-sphere $\mathbb{S}^{N-1} \subset \mathbb{R}^N$.

By choosing a set of rules such as Gram-Schmidt orthonormalization of the canonical basis in $\mathbb{R}^{N \times N}$, it is possible to analytically compute from $X$ an orthogonal complement $X^\perp \in \mathbb{R}^{N \times (N - K)}$, such that $\tilde{X}^\top \tilde{X} = I_N$, where $\tilde{X} = \begin{bmatrix}X & X^\perp \end{bmatrix} \in O(N)$.

Note that if we differentiate both sides of the constraint $X^\top X = I_K$, we get $\dv{X}^\top X + X^\top \dv{X} = 0$, which leads to $(X^\top \dv{X})^\top = - X^\top \dv{X}$.
Therefore, $X^\top \dv{X}$ is a $K \times K$ skew-symmetric matrix with $\frac{K(K-1)}{2}$ unique entries in the strict upper triangle.

Densities of distributions on the Stiefel manifold are usually written with respect to the invariant metric on the Stiefel manifold, which we will define separately for the two cases $K=N$ and $K < N$.

For $X \in O(N)$ with the $i$th column written $x_i$, we define the invariant measure as the exterior product of the unique elements of this skew-symmetric matrix:
\[
  (X^\top \dv{X})^\wedge \equiv \bigwedge_{j=1}^N \bigwedge_{i=j+1}^N (X^\top \dv{X})_{ij} = \bigwedge_{j=1}^N \bigwedge_{i=j+1}^N x_i^\top \dv{x}_j.
\]
This measure is called invariant because if $f(X) = A X B$ for $A,B \in O(N)$ and $\mathcal{S} \subset O(N)$ then $\int_{\mathcal{S}} (X^\top \dv{X}) = \int_{f(\mathcal{S})} (X^\top \dv{X})$.
It is also called the Haar measure on $O(N)$ \cite{muirhead2009aspects}..

To choose an invariant measure for $K < N$, we first choose an $X^\perp$ analytically computed from $X$.
Because $X^\perp$ is analytically computed, $\dv{X^\perp_{ij}} = 0$.
We can then construct the invariant measure for $K<N$ from the invariant measure for $O(N)$:

\[
  (X^\top \dv{X})^\wedge \equiv \bigwedge_{j=1}^N \bigwedge_{i=j+1}^N x_i^\top \dv{x}_j = \bigwedge_{j=1}^K \bigwedge_{i=j+1}^N x_i^\top \dv{x}_j,
\]

Though we needed to choose a $X^\perp$, the invariant measure does not depend on the choice of $X^\perp$.
Moreover, this measure is invariant to left-actions by $O(N)$ and right-actions by $O(K)$.
For more details, see \cite{muirhead2009aspects}.

When $K=1$, the invariant measure is equivalent to the usual Hausdorff measure on the sphere.

\subsubsection{Givens parameterization of Stiefel manifold}

\subsubsection{Cayley parameterization of Stiefel manifold}

\subsubsection{Householder parameterization of Stiefel manifold}

\subsubsection{Matrix exponential parameterization of Orthogonal group}

\subsection{Positive definite matrices}

[SA: all transformations below can be generalized to SPD matrices of fixed rank, but we'd need to know what the common reference measure is for densities of SPD matrices, and I can't recall seeing one.]

An $N \times N$ positive definite (PD) matrix $X$ is a symmetric matrix that satisfies $v^\top X v \ge 0$ for all non-zero vectors $v$.
Covariance matrices are examples of PD matrices.

Density functions for distributions of PD matrices, such as the Wishart distribution, are generally defined with respect to the Lebesgue measure of the upper triangular elements $(\dv{X})^\wedge$.

\subsubsection{Cholesky parameterization}

Given a positive (semi-)definite $X$ of rank $K$, we can uniquely decompose it as $X = Z^\top Z$ where $Z$ is an $N \times N$ upper triangular matrix where $Z_{ii} > 0$ for all $i \le K$ and $Z_{ij} = 0$ for all $i > K$.
$Z$ has $\frac{K(K+1)}{2} + (N - K)K$ degrees of freedom.
When $K=N$, $X$ is positive-definite.

We can construct $X$ by composing 2 bijective maps.
The first, $f_1$, enforces the constraint of the positive diagonal:

\[Z_{ij} = f_1(Y)_{ij} = \begin{cases} \exp(Y_{ij}) & \text{ if } i = j \le K \\ Y_{ij} & \text{otherwise} \end{cases}\]

The density correction for this map is $\absdet{J_{f_1}(Y)} = \exp(\sum_{i=1}^K Y_{ii})$.

The second map is $f_2: Z \mapsto Z^\top Z = X$.
To compute the Jacobian of the map, we use the half-vectorization $\vectu(X)$.
Note that $X_{ij} = \sum_{k=1}^{\min(i, j)} Z_{ki} Z_{kj}$.
$X_{ij}$ is thus only dependent on elements of $Z$ that are above and to the left of the $(i, j)$ position.
As a result, the Jacobian of $\vectu \circ f_2 \circ \\vectu^{-1}$ is lower-triangular, and its determinant is only the product of its diagonal elements: $\absdet{J_{f_2}} = \prod_{1 \le i \le j \le N} \frac{\partial X_{ij}}{\partial Z_{ij}}$.

In this index range, $\frac{\partial X_{ij}}{\partial Z_{ij}} = (1 + \delta_{ij}) Z_{ii}$.
As a result

\[
  \absdet{J_{f_2}(Z)} = 2^N \prod_{j = 1}^N \prod_{i = 1}^j Z_{ii} = 2^N \prod_{i=1}^N |Z_{ii}|^{N - i + 1} \\
\]

Combining the two bijective maps as $f = f_2 \circ f_1$, the combined density correction is

\[
  \absdet{J_{f}(Y)} = 2^N \exp\left( \sum_{i=1}^N (N - i + 2) Y_{ii}\right).
\]

\subsubsection{Matrix exponential parameterization}

For every PD matrix $X \in \mathbb{R}^{N \times N}$, there is a unique symmetric matrix $Y$ such that $X = \Exp(Y)$, where $\operatorname{Exp}(\cdot)$ is the matrix exponential, defined as
\[\Exp(Y) = \sum_{k=0}^\infty \frac{1}{k!} Y^k.\]

Let $Y = U \Lambda U^\top$ be the symmetric eigendecomposition of $Y$, where the matrix of eigenvectors $U$ is orthogonal, and the $\Lambda = \operatorname{diag}(\lambda)$ for $\lambda \in \mathbb{R}^N$.
Then $Y^2 = U \Lambda U^\top U \Lambda U^\top = U \Lambda^2 U^\top$.
In fact, $Y^c = U \Lambda^c U^\top$ for all real $c$.
As a result,
\[\Exp(Y) = U \Exp(\Lambda) U^\top,\]
where $\Exp(\Lambda)_{ij} = e^{\lambda_i} \delta_{ij}$.

$\Exp(Y)$ is one example of a matrix function $f$, which we can roughly define for symmetric matrices as any function that obeys
\[f(Y) = U f(\Lambda) U^\top,\]
where $f(\Lambda)_{ij} = f(\lambda_i)\delta_{ij}$.
\footnote{
  In a similar way, we can generalize other scalar functions like the inverse, trigonometric, and hyperbolic functions to functions of matrices.
  All results in this section apply to these functions as well.
}

For such functions $f$, we can compute a common form of the Jacobian determinant.
Differentiating the expression, we find
\[
\begin{aligned}
  \dv{(f(Y))} &= \dv{U} f(\Lambda) U^\top + U f'(\Lambda) \dv{\Lambda} U^\top + U f(\Lambda) \dv{U}^\top\\
  U^\top \dv{(f(Y))} U &= U^\top \dv{U} f(\Lambda) + f(\Lambda) \dv{U}^\top U + f'(\Lambda)\dv{\Lambda},
\end{aligned}
\]
where $f'(\Lambda)_{ij} = \frac{\dv{(f(\lambda_i))}}{\dv{\lambda_i}}$.

Recall that $U^\top \dv{U}$ is skew-symmetric.
Note also that for skew-symmetrix $Z = -Z^\top$ and diagonal matrix $\operatorname{diag}(v)$,
\[(Z\operatorname{diag}(v) + \operatorname{diag}(v)Z^\top)_{ij} = Z_{ij} v_j + v_i Z_{ji} = Z_{ij} v_j - v_i Z_{ij} = Z_{ij} (v_j - v_i) = Z \circ V,\]
where $V_{ij} = v_j - v_i$ and $\circ$ is the Hadamard (elementwise) product.

From the right-hand side, note then that
\[(U^\top \dv{(f(Y))} U)_{ij} = \begin{cases} (U^\top \dv{U})_{ij} (f(\lambda_j) - f(\lambda_i)) & \text{ if } i \ne j \\ f'(\lambda_i) \dv{f(\lambda_i)} &\text{ if } i = j \end{cases}\]

The exterior product of the diagonal part is
\[
  \bigwedge_{i=1}^N f'(\lambda_i) \dv{\lambda_i} = \left(\prod_{i=1}^N f'(\lambda_i)\right) (\dv{\Lambda})^\wedge,
\]
while the exterior product of the super-diagonal part is
\[
  \bigwedge_{j=1}^N \bigwedge_{i=j+1}^N (U^\top \dv{U})_{ij} (f(\lambda_j) - f(\lambda_i)) = \left( \prod_{j=1}^N \prod_{i=1}^{j-1} (f(\lambda_j) - f(\lambda_i)) \right) \left(U^\top \dv{U} \right)^\wedge
\]

Lastly, note for any symmetric $Z$ and orthogonal matrix $U$,
\[(U^\top \dv{Z} U)^\wedge = \det(U^\top)^N \det(U)^N (\dv{Z})^\wedge = (\dv{Z})^\wedge.\]

As a result, we can compute $(\dv(f(Y)))^\wedge$ by taking the exterior product of the super-diagonal and diagonal parts: 
\[(\dv{(f(Y))})^\wedge = \left(\prod_{j=1}^N f'(\lambda_j) \prod_{i=1}^{j-1} (f(\lambda_j) - f(\lambda_i))\right) ((U^\top \dv{U})^\wedge \wedge \dv{\Lambda})\]

If $f$ is the identity map $f(Y) = Y$, then
\[ (\dv{Y})^\wedge = \left(\prod_{j=1}^N \prod_{i=1}^{j-1} (\lambda_j - \lambda_i)\right) \left((U^\top \dv{U})^\wedge \wedge \dv{\Lambda}\right)\]

Combining the latter two expressions:
\[(\dv{(f(Y))})^\wedge = \left(\prod_{j=1}^N f'(\lambda_j) \prod_{i=1}^{j-1} \frac{f(\lambda_j) - f(\lambda_i)}{\lambda_j - \lambda_i}\right) (\dv{Y})^\wedge\]

[REF, Edelman's notes] previously found this result.

For $X = f(Y) = \Exp(Y)$ then,
\[ (\dv{X})^\wedge = \left(\prod_{j=1}^N e^{\lambda_j} \prod_{i=1}^{j-1} \frac{e^{\lambda_j} - e^{\lambda_i}}{\lambda_j - \lambda_i}\right) (\dv{Y})^\wedge,\]

so the Jacobian determinant of the exponential map is $\absdet{J_{\Exp}(Y)} = \prod_{j=1}^N e^{\lambda_j} \prod_{i=1}^{j-1} \left|\frac{e^{\lambda_j} - e^{\lambda_i}}{\lambda_j - \lambda_i}\right|$.

Note that this expression is undefined when any eigenvalues are repeated.
This is not a problem for MCMC, since the eigenvalues are non-unique with probability 0, but non-unique eigenvalues could occur when maximizing a log-density.

Because $\lim_{h \to 0} \frac{f(x) - f(x + h)}{h} = f'(x)$, we can then modify the expression to handle such cases:
\[
  \absdet{J_{\Exp}(Y)} = \prod_{j=1}^N \prod_{i=1}^j \begin{cases}
    \left|\frac{e^{\lambda_j} - e^{\lambda_i}}{\lambda_j - \lambda_i}\right| &\text{ if } \lambda_i \ne \lambda_j \\ 
    e^{\lambda_i} &\text{ if } \lambda_i = \lambda_j
  \end{cases}.
\]

While the cost of computing the Jacobian determinant is $O(N^2)$, the cost of computing the eigendecomposition naively is $O(N^3)$, so this transformation is expensive.
An attractive alternative would be to define the transform directly from the eigendecomposition of $X$.

\subsubsection{Eigendecomposition parameterization}

To define a transform directly from the symmetric eigendecomposition of $X$, we first need to note that the eigendecomposition is not unique; if we swap any two columns $i$ and $j$ of $U$ as well as swapping the eigenvalues $\lambda_i$ and $\lambda_j$, the result is the same matrix.

When the eigenvalues are all unique, we can make the eigendecomposition also unique by enforcing the ordering of eigenvalues $\lambda_i > \lambda_j > 0$ for $j > i$.
This is the ordering generally used for symmetric singular value decomposition.

So we can construct a bijective map with the maps $f_1$ that produces an ordered vector of positive eigenvalues and $f_2$ that produces an orthogonal matrix.
Consider $\lambda = f_1(y_1)$ and $U = f_2(y_2)$.

Let $f_1$ be the map that computes each $\lambda_i$ as
\[
  \lambda_i = \begin{cases}
    e^{(y_1)_N} &\text{ if } i = N\\
    \lambda_{i-1} + e^{(y_1)_i} &\text{ if } 1 \le i < N
  \end{cases}
\]
The Jacobian determinant of this map is $\absdet{J_{f_1}(y_1)} = \exp\left(\sum_{i=1}^N (y_1)_i\right)$.

We can then choose for $f_2$ any of the transforms to the orthogonal matrices previously discussed.
The combined Jacobian determinant of $f: (y_1, y_2) \mapsto f_2(y_2) \operatorname{diag}(f_1(y_1)) f_2(y_2)^\top = X$ is then 
\[ \absdet{J_{f}(y_1, y_2)} = \absdet{J_{f_2}(y_2)} \exp\left(\sum_{i=1}^N (y_1)_i\right).\]

\subsubsection*{Acknowledgements}

We would like to thank \url{matrixcalculus.org} for providing an
easy-to-use symbolic matrix derivative calculator.



\bibliography{all}{}
\bibliographystyle{plain}

\end{document}
