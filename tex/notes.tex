Sampling algorithms play an important role in Statistics, in this case
we refer to drawing samples from a probability density or distribution
(not survey sampling). Monte Carlo Markov chain methods are commonly
used for high dimensional sampling. In this paper, we look at
Hamiltonian Monte Carlo, which uses hamiltonian dynamics to propose
samples for the Metropolis algorithm. Bayesian Inference often uses
parameters with constraints, but HMC optimizes on the unconstrained
space. It is difficult to use Monte Carlo estimation on a constrained
domain ~\cite{neal2008optimal}

The softmax function can be understood from Multinomial Logistic
Regression employed for predicting probabilities of a Categorically
distributed variable. Geometrically it maps $R^K$ to the boundary of a
unit K-simplex(it is simply the convex hull of $k+1$ affinely
independent points in $R^K$. Essentially it transforms a vector of
size K to another vector of size K where the outputs sum to 1. It is
worth noting that the mapping is actually from $R^K$ to $R^{K-1}$, so
when a vector of size $K$ is transformed the $K_th$ vector is simply
$1$- sum of k-1 vectors

\subsection{Simplex softmax parameterization}

Let $\Delta^n$ indicate the unit $n$-simplex with $n$ positive
elements that sum to 1 and $\Delta^n_-$ indicate the first $n-1$
elements, from which the final element can be uniquely determined.

We define the transformation
$\phi: \mathbb{R}^{n-1} \to \Delta^n_-: y \mapsto x_-$, where
$x=\begin{pmatrix}x_- \\ \frac{1}{r}\end{pmatrix} \in \Delta^n$,
$x_i = \frac{1}{r} e^{y_i}$ for $1 \le i \le n-1$, and
$r = 1 + \sum_{i=1}^{n-1} e^{y_i}$.

First we compute the scalar derivatives:
\[
\begin{aligned}
  \frac{\mathrm{d} r}{\mathrm{d} y_j}
  &= e^{y_j} = r x_j\\
  \frac{\mathrm{d} x_i}{\mathrm{d} y_j}
  &= \delta_{ij} \frac{1}{r} e^{y_i} - \frac{1}{r^2} e^{y_i}
  \frac{\mathrm{d} r}{\mathrm{d} y_j}
  = \delta_{ij} x_i - x_i x_j, \quad 1 \le i \le n-1
\end{aligned}
\]
where
$\delta_{ij} = \begin{cases} 1 &\text{if } i = j \\ 0
  &\text{otherwise}\end{cases}$ is the Kronecker delta.

If $\mathrm{diag}(x)$ is the diagonal matrix whose diagonal are the
elements of $x$, then the Jacobian is
\[
  J = (I_{n-1} - x_- \boldsymbol{1}_{n-1}^\top) \operatorname{diag}(x_-),
\]
where $\boldsymbol{1}_n$ is the $n$-vector of ones.

Using Sylvester's determinant theorem,
$|I_{n-1} - x_- \boldsymbol{1}_{n-1}^\top| = 1 -
\boldsymbol{1}_{n-1}^\top x_- = 1 - \sum_{i=1}^{n-1} x_i = x_n$, so
$$\mathrm{correction} = |J| = x_n \prod_{i=1}^{n-1} x_i = \prod_{i=1}^{n} x_i = \exp\left(\sum_{i=1}^{n-1} y_i\right) \left(1 + \sum_{i=1}^{n-1} e^{y_i}\right)^{-n}$$



\subsubsection{Softmax Transform}

To calculate the determinant of the Jacobian of the inverse transform,
we start by noting that $s = \textrm{exp} \circ \textrm{norm}$, where
$\textrm{exp}$ is the elementwise exponential function and
\textrm{norm} is defined by
\[
  \textrm{norm}(z) = \frac{z}{\textrm{sum}(z) + 1}.
\]
As such, the resulting Jacobian determinant is the product of the
Jacobian determinants of the component functions,
\[
  \absdet{J_s(y)}
  = \absdet{J_{\textrm{exp}}(y)} \absdet{J_{\textrm{norm}}(z)},
\]
where $z = \textrm{exp}(y)$.  The Jacobian for the exponential
function is diagonal, so the determinant is the product of the
diagonal of the Jacobian, which for $y \in \mathbb{R}^{N-1}$ is
\[
  \absdet{J_{\textrm{exp}}(y)} = \textrm{prod}(\exp(y)).
\]
As above, let $z = \exp(y) \in (0, \inf)^{N-1}$.  We can differentiate
$\textrm{norm}$ to derive the Jacobian,
\[
  J_{\textrm{norm}}
  = \frac{1}{1 + \textrm{sum}(z)} \mathbb{I}_{N-1}
  - \left(\frac{1}{(1 + \textrm{sum}(z))^2} \beta \right)
  \textrm{vector}_{N-1}(1)^{\top},
\]
where $\mathbb{I}_{N-1}$ is the $(N - 1) \times (N - 1)$ unit matrix and
$\textrm{vector}_{N-1}(1)$ is the $N - 1$-vector with values 1.  Using
the matrix determinant lemma,\footnote{The matrix determinant lemma
  is \[\textrm{det}(A + u v^{\top}) = (1 + v^{\top} A^{-1} u)
    \textrm{det}(A).\]}
we have
\begin{eqnarray*}
  \textrm{absdet}(J_{\textrm{norm}}(z))
  & = &
  \left(
    1
    + \textrm{vector}_{N-1}(1)^{\top}
    \left(\frac{1}{1 + \textrm{sum}(z)} \mathbb{I} \right)^{-1}
    \frac{-z}{(1 + \textrm{sum}(z))^2}
    \right)
    \ \textrm{det}\left(\frac{1}{1 + \textrm{sum}(z)} \mathbb{I}
        \right)
  \\[6pt]
  & = &
  \left(
    1 
    + \textrm{sum}\left( \frac{-(1 + \textrm{sum}(z)) z}{(1 +
        \textrm{sum}(z))^2} \right)
  \right)
        \ \left( \frac{1}{1 + \textrm{sum}(z)} \right)^{N-1}
  \\[6pt]
  & = &
        \left(1 + \textrm{sum}\left(\frac{-z}{1 + \textrm{sum}(z)} \right)\right)        
        \ \left( \frac{1}{1 + \textrm{sum}(z)} \right)^{N-1}
  \\[6pt]
  & = & \left( 1 - \textrm{sum}(\textrm{norm}(z)) \right) 
        \ \left( \frac{1}{1 + \textrm{sum}(z)} \right)^{N-1}
  \\[6pt]
  & = & \left( \frac{1}{1 + \textrm{sum}(z)} \right)^N.
\end{eqnarray*}
Thus the entire absolute determinant of the Jacobian is defined by the
product, 
\[
  \absdet{J_s(y)}
  \ = \
  \textrm{prod}(\exp(y))
  \, \left( \frac{1}{1 + \textrm{sum}(\exp(y))} \right)^N.
\]
and our final expression for densities for unconstrained $y \in
\mathbb{R}^{N-1}$ is
\[
  p_Y(y)
  = p_X(\textrm{alr}^{-1}(y))
  \, \textrm{prod}(\exp(y))
  \left( \frac{1}{1 + \textrm{sum}(\exp(y))} \right)^N
\]  




